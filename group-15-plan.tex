%% Template for SDP report, adapted from mlp_cw2_template, 2018. 

%% Based on  LaTeX template for ICML 2017 - example_paper.tex at 
%%  https://2017.icml.cc/Conferences/2017/StyleAuthorInstructions

\documentclass{article}
\usepackage[T1]{fontenc}
\usepackage{amssymb,amsmath}
\usepackage{txfonts}
\usepackage{microtype}
\usepackage{xspace}
\xspaceaddexceptions{\%}

% Lists with less spacing between items
\usepackage{paralist}

% For figures
\usepackage{graphicx}
\usepackage{subfig} 

% For citations
\usepackage{natbib}

% For algorithms
\usepackage{algorithm}
\usepackage{algorithmic}

% the hyperref package is used to produce hyperlinks in the
% resulting PDF.  If this breaks your system, please commend out the
% following usepackage line and replace \usepackage{mlp2017} with
% \usepackage[nohyperref]{mlp2017} below.
\usepackage{hyperref}
\usepackage{url}
\urlstyle{same}

% Packages hyperref and algorithmic misbehave sometimes.  We can fix
% this with the following command.
\newcommand{\theHalgorithm}{\arabic{algorithm}}


% Set up MLP coursework style (based on ICML style)
\usepackage{mlp2018}
\mlptitlerunning{SDP Demo \demoNumber  Group (\groupNumber)}
\bibliographystyle{icml2017}


\DeclareMathOperator{\softmax}{softmax}
\DeclareMathOperator{\sigmoid}{sigmoid}
\DeclareMathOperator{\sgn}{sgn}
\DeclareMathOperator{\relu}{relu}
\DeclareMathOperator{\lrelu}{lrelu}
\DeclareMathOperator{\elu}{elu}
\DeclareMathOperator{\selu}{selu}
\DeclareMathOperator{\maxout}{maxout}







%% You probably do not need to change anything above this comment

%% REPLACE the details in the following commands with your details
\setGroupNumber{15}
\setGroupName{Detroit}
\setProductName{Tadashi}
\setLogoFileName{figs/logo-small.png}

\begin{document} 

\makeSDPTitle{Project Plan}

% Previous MLP Style Title Layout working. 
% \twocolumn[
    % \mlptitle{\productName: SDP Demo \demoNumber}
    % \centerline{Group \groupNumber: \groupName}
    % ]

\begin{abstract}
  We propose an assistive healthcare robot, {\it Tadashi}, to automate simple tasks within a care home or supported living environment and allow nurses to spend more time caring for their patients.
  {\it Tadashi} will automate three key tasks in the nurse's day. Firstly, waking a patient up at a time specified by the nurse, by coming into their room and speaking to them. Secondly, bringing water and food to patients at specified times or on request. Thirdly, checking on the welfare of the patient while the nurse is occupied elsewhere, by coming to their room and asking the patient if they are okay and if they need a nurse to attend to them. {\it Tadashi} will be able to move autonomously around a prototype living space using input from sensors, initially overhead cameras then sensors integrated into the robot itself. 
\end{abstract} 

\section{Goal description}
Nurses are overworked: they spend a large amount of time on menial tasks, limiting the amount of time they can spend caring for patients. Our assistive healthcare robot allows nurses to automate certain menial and administrative tasks, allowing them to spend less time on the menial work, and more time doing what matters: caring for their patients. 

\subsection{Relevance of the system}
\subsubsection{The problem space}
Tadashi is designed to tackle two problems facing nurses in assisted living situations:
\begin{enumerate}
\item High nurse:patient ratios (ie. many patients to one nurse), meaning nurses have little time to dedicate per patient;
\item High administrative loads on nurses, meaning they must spend more time on administration and menial tasks and less on caring for their patients directly.
\end{enumerate}

There is no national guidance on staffing levels and nurse:patient ratios in care homes \cite{rcnstaffingadvice}, but research shows that in care homes there is an average ratio of 18 patients per registered nurse during the day, and 26 patients at night \cite{rcnstaffingguidance}.

With this many patients to take care of, nurses struggle to get the time they need with patients. In a 2017 survey on the impact of high nurse:patient ratios \cite{unison}, 63.2\% of nurses said that comforting or talking to patients was rushed, unfinished, not done to an acceptable standard, or missed entirely.

Equally, a 2013 survey showed that nurses spend almost 1/5 of their day on administrative tasks; and only 20\% are satisfied with how they spend their time --- preferring to spend less time on admin and more time on direct care \cite{rcnpol}.

By automating some tasks using Tadashi, nurses can decrease the amount of time they spend on administrative tasks, and increase the amount of time they spend caring for patients.

With elderly populations in particular, hydration and nutrition presents a pressing concern \cite{hydrate}. A Care Quality Commission inspection in 2011 found that three out of 12 NHS trusts inspected failed to meet standards required by law in meeting patient needs regarding nutrition and hydration \cite{cqc}. A 2018 report from the Office for National Statistics found that more than 600 care home residents died suffering from malnutrition or dehydration between 2013 and 2017 \cite{ons}.

Tadashi will allow for delivery of water or food to patients at a specified time --- or patients to request these deliveries themselves --- in order to ensure patients have access to regular food and water. Additionally, by tracking these requests we will provide a dashboard for nurses to view when patients last had food or water: this provides added value to nurses along with reducing their workload through automation. 

The problems mentioned above will only get worse in the future. It is estimated that by 2035, up to 190,000 more people aged 65 years or above will require some level of care, and increase of 86\% from today \cite{lancet}. Meanwhile, ongoing staffing issues in the NHS mean that in 10 years time the NHS will have a shortfall of 108,000 nurses \cite{nuffield}. This combination of factors will drive demand for innovative solutions --- including assistive technology like Tadashi.


\subsubsection{Existing solutions}
The most relevant existing solution to the problems we have identified is the work of Fraunhofer IPA on service robots in residential care facilities. As part of the WiMi-Care project, they implemented and tested Care-O-bot 3, a `robot butler' that tracks residents' hydration and brings them water if they have not drunk enough. The goal of this project was to automate certain service-related tasks in order to relieve pressure on care staff \cite{fraunhofer}.

The key takeaways from this work that we will take into account in our project:
\begin{itemize}
\item Patient feedback to the robot was positive: ``inhabitants  understood  the  idea  of  a  robot  supporting  the  staff  without replacing them and showed no fear to interact with the machine'' \cite{springer}. 
\item Adding speech output to address patients by name helped to improve perceptions of the robot and compliance with drinking the water it offered \cite{ieee}. 
\item Staff reacted positively to the introduction of the robot: ``The overall reaction from the personnel ... was very positive'' (ibid). 
\end{itemize}

\subsection{High-level description} 
We describe the following user stories to exemplify the functionality of the system: 
\begin{enumerate}
\item Tadashi waking the patient
\item Tadashi bringing food or water to the patient
\item Tadashi checking on the patient
\item Nurse viewing patient nutrition / hydration on an app
\end{enumerate}

\subsubsection{Waking the patient}
\begin{enumerate}
\item The nurse specifies in the app what time each patient should wake up.
\item At the specified time, Tadashi navigates to the specified patient's room.
\item Once in the room, Tadashi speaks to wake the user up. A button is accessible to the patient to press once they have woken up:
  \begin{enumerate} 
  \item If the patient does not press the button, Tadashi sends an alert to the nurse's app, who can then go check on the person. 
  \end{enumerate}
\item Tadashi returns to his starting spot to await the next command. 
\end{enumerate}

\subsubsection{Bringing food or water to the patient}
\begin{enumerate}
\item Either the nurse or the patient specifies a need for food or water:
  \begin{itemize}
  \item The nurse specifies in the app for each patient:
    \begin{itemize}
    \item The intervals at which they should be brought water (or another drink).
    \item The times at which they should be brought food.
    \end{itemize}
  \item The patient presses a button on their controller to request Tadashi brings them food or water. 
  \end{itemize}
\item Tadashi navigates to the pick-up point for food or water.
\item Tadashi moves to the patient's room and passes over the food or water using its arm.
  \begin{itemize}
  \item If the patient is at a different height to the robot, Tadashi will increase his height using a lift to get to an appropriate height for the patient and deliver food or water without spillage. 
  \end{itemize}
\item Tadashi returns to his starting spot to await the next command.
\end{enumerate}


\subsubsection{Checking on the patient}
\begin{enumerate}
\item The nurse specifies in the app that Tadashi should go check on a specified patient. 
\item Tadashi navigates to the specified patient's room.
\item Once in the room, Tadashi speaks to the user and asks if they are okay:
  \begin{enumerate}
  \item A button is accessible to the patient to press to reply yes or no.
  \item If the user presses the okay button, Tadashi informs the nurse through the app. 
  \item If the user presses the not okay button, Tadashi informs the nurse through the app with an urgent alert.
    \begin{itemize}
    \item The patient can alternatively, when Tadashi is not present, request the nurse's help by pressing a dedicated button on their controller. 
    \end{itemize}
  \end{enumerate}
\item Tadashi returns to his starting spot to await the next command. 
\end{enumerate}


\subsubsection{Checking patient nutrition and hydration}
\begin{enumerate}
\item The nurse opens the app to `dashboard mode'.
\item The app displays for each patient a history of their nutrition and hydration deliveries. 
\end{enumerate}
\section{Task planning}

\subsection{Milestones}
By the final demo, the robot will have the following key functionality:
\begin{itemize}
\item Free movement of the robot to patient rooms;
\item Steady movement of a mechanical arm along a horizontal plane, allowing for delivery of food and water: the arm will have a bucket or container attached to it to allow for transportation without needing to consider gripping mechanisms;
\item Steady movement of a lift up and down, allowing for the robot to be level with the patient (figure~\ref{fig:lift});
\item Ability to move to a room on request and on a timetable;
\item Ability to send and receive messages to and from the app. 
\end{itemize}


\begin{figure}[tb]
\vskip 5mm
\begin{center}
\centerline{\includegraphics[width=\columnwidth]{figs/lift}}
\caption{Proposed scissor lift using Lego, allowing the robot to adjust its height to the patient. Image source: \href{https://education.lego.com/en-us/lessons/pneumatics/scissor-lift}{Lego Education}}
\label{fig:lift}
\end{center}
\vskip -5mm
\end{figure}

Likewise, the app will have the following functionality to support the robot:
\begin{itemize}
\item A usable and accessible Android app, supported by user testing;
\item Ability to send and receive messages to and from the robot;
\item Ability to create calendars and timetables for when patients should be woken up and given food and water;
\item A visualization dashboard for patient hydration and nutrition, showing their history of food and water deliveries;
\item GDPR compliance to protect patient privacy, including a privacy policy and data encryption at rest and in transit. 
\end{itemize}

We have split our team into three groups to manage the respective tasks of robot building, robot programming and app development. The work for each team is designated by four milestones, each corresponding to the Sunday before the Wednesday demo date. In the following sections we detail what we expect to have achieved by each milestone. 

\subsubsection{Milestone 1 (February 2nd)}
For the first milestone we build and test basic components, from which we will build on throughout the rest of the course. 

The robot building team will present their prototype design, choosing from either TurtleBot or Lego: a minimal chassis that can hold loads of at least 500 grams, move forwards and backwards, and make turns.

The robot software team will demonstrate basic movement control, where the robot can move as specified by external commands. They will also test and demonstrate initial inputs from sensors (overhead cameras and sensors built into the robot), to be used later to locate the robot within the demo space.

The app building team will demonstrate a skeleton Android app running on a device, with basic UI design for activity screens; as well as demonstrate a database set up in Firebase interacting with the Android app. 

\subsubsection{Milestone 2 (February 23rd)}
At this milestone we begin to integrate each team's work together, building the connections necessary to connect between each component of the project. 

The robot building team will finalize the design of the chassis, and demonstrate basic component integration, showing all components controllable by command:
\begin{itemize}
\item An arm operating on a horizontal plane to deliver food and water to the patient;
\item A scissor lift able to raise the robot up and down;
\item Buttons and sensors integrated with the hardware.
\end{itemize}

The robot software team will demonstrate more advanced movement control: being able to get the layout of the demo space, recognize the robot's location and orientation, and locate the target to move to. 

The app building team will demonstrate the ability to receive alerts from the robot or patient controller. 

\subsubsection{Milestone 3 (March 8th)}
At this milestone we flesh out key functionality within each component, building out APIs and beginning to fully connect all components. 

The robot building team will integrate 3D printed buttons and finalize any component choices, as well as demonstrating smooth movement of the arm and lift. They will also create a prototype for the patient controller. 

The robot software team will demonstrate functional movement of the robot around the space using overhead and built-in sensors, possibly avoiding obstacles. 

The app building team will demonstrate key functionality within the app, allowing for:
\begin{itemize}
\item Timetabling of food/water delivery;
\item Timetabling of wake-up time;
\item Requesting the robot to check-in on a patient.
\end{itemize}
The app will also demonstrate GDPR compliance with a written privacy policy and encrypted data at rest and during transit.

\subsubsection{Milestone 4 (March 29th)}
By this milestone we are ready to demonstrate a functioning robot, connecting the work of the robot hardware and software teams and the app team. We will be able to demonstrate each of the use-cases in practice, with full integration between components.

The app team will demonstrate quantitative and qualitative analysis of UI usability testing, and any steps made to improve the UI based on this testing.

\subsection{Task Decomposition}

For each milestone we have assigned sub-tasks to the relevant team, and also planned how our sub teams will work together to implement these as working features.

For each sub-task we have assigned a difficulty rating ranging from simplest to most complex with XS, S, M, L, XL. We assign these values as estimations of the time and difficulty involved with implementing the sub-task, based on initial research, expert feedback, and gut feeling. 


\subsubsection{Robot building}
{\bf Milestone 1}

Decide if using Turtlebot or Lego:
\begin{itemize}
\item Build Lego and Turtlebot robots (M)
\item Implement basic movement ability (move forward / backward, turn) (M)
\item Test each robot on battery life, turn radius, maximum load, speed (L)
\item Based on tests, choose Turtlebot or Lego (M)
\end{itemize}

{\bf Milestone 2}

Finalize design of robot and component integration:
\begin{itemize}
\item Build and add arm operating on one plane, intended to deliver food and water to patient (M)
\item Build and add scissor lift to move robot up and down (L)
\item Integrate buttons and sensor with hardware, working with robot software team (L)
\item Allow components to be controlled by command (M)
\end{itemize}

{\bf Milestone 3}

Integrate 3D printed buttons for the users to press:
\begin{itemize}
\item Design and manufacture buttons for 3D printing (L)
\end{itemize}

Create and integrate patient controller:
\begin{itemize}
  \item Design and manufacture basic controller (L)
  \item Interface controller using WiFi or Bluetooth with robot or app (M)
\end{itemize}

Finalize component choices based on robot software testing:
\begin{itemize}
\item Ensure smooth movement of arm on horizontal plane (M)
\item Ensure smooth movement of lift on vertical plane (M)
\end{itemize}


{\bf Milestone 4}

Work with software and app teams to finalize connectivity (XL)


\subsubsection{Robot programming}

{\bf Milestone 1}

Demonstrate basic movement control:
\begin{itemize}
\item Implement moving set distances and turning (S)
\end{itemize}

Demonstrate input from sensors:
\begin{itemize}
\item Test to see if we can get inputs from overhead and internal sensors (M)
\end{itemize}

{\bf Milestone 2}

Demonstrate advanced movement control:
\begin{itemize}
\item Recognize robot location in demo space using overhead cameras (M)
\item Recognize robot orientation using internal sensors (S)
\item Locate target in demo space (L)
\item Store internal representation of demo space (L)
\end{itemize}

{\bf Milestone 3}

Demonstrate functional movement of the robot:
\begin{itemize}
\item Independent navigation of the space using sensor inputs (overhead cameras / internal sensors); allowing the robot to move to specific locations (XL)
\end{itemize}

Establish connectivity with the Android app:
\begin{itemize}
\item Work with app team to research and determine approach to connectivity (M)
\item Add software to send / receive messages (L)
\end{itemize}

Establish connectivity with patient controller:
\begin{itemize}
\item Work with hardware team to receive alerts from patient controller (M)
\end{itemize}

{\bf Milestone 4}

Work with hardware and app teams to finalize connectivity (XL)


\subsubsection{App development}

{\bf Milestone 1}

Demonstrate skeleton Android app:
\begin{itemize}
\item Start an Android app project in Kotlin (S)
\item Create 5 activity screens (M)
\end{itemize}

Demonstrate basic database:
\begin{itemize}
\item Plan out and set up database schema (S)
\item Implement database using Firebase (M)
\item Show ability to connect to database (send / receive information) through app (L)
\end{itemize}

{\bf Milestone 2}

Demonstrate connectivity:
\begin{itemize}
\item Integrate with robot software team to build communication API (L)
\item Show alert from robot within the app (M)
\end{itemize}

{\bf Milestone 3}

Implement key app functionality:
\begin{itemize}
\item Timetabling of food / water delivery (M)
\item Timetabling of wake-up time (M)
\item Ad-hoc `check-in' requests (M)
\item Dashboard to see hydration and nutrition history (M)
\end{itemize}

Encrypt user data at rest and in transit:
\begin{itemize}
\item Investigate encryption options for Firebase (M)
\item Use HTTPS connections when communicating over WiFi (M)
\end{itemize}

{\bf Milestone 4}

Work with hardware and software teams to finalize connectivity (XL)

Test UI accessibility:
\begin{itemize}
\item Set up ethical approval, if necessary (S)
\item Research accessibility principles (M)
\item Find survey participants (M)
\item Test accessibility of UI based on metrics found in research (L)
\item Make UI changes based on feedback from survey (M)
\end{itemize}

\subsection{Resource distribution}

The 200 hours per member will include the following for every member: {\bf at least} 15-20 hours worth of team meetings, 4 hours of demos with another 4 for preparation, 2 hours of workshops with each member attending at least one, and 1-2 hours for the individual reflection report. This leaves each member with 168 hours to dedicate to their work; roughly 1500 hours in total, or 500 hours per team. Each milestone has the same deadline across the teams - the date of the corresponding demo. They are shown in (Table~\ref{tab:demo-dates}). (Table~\ref{tab:rb-rd}) shows the resource distribution of the robot building teams 500 hours in terms of the equipment, skills, and estimated hours each subtask of the milestones will require. Similarly, (table~\ref{tab:rp-rd}) shows the robot coding teams resource distribution, and (table~\ref{tab:app-rd}) shows this for the app team.
\begin{table}[]
  \begin{center}
  \begin{tabular}{ll}
    \hline
    Demo & Date   \\
    \hline
    1 & Feb 5 \\
    2 & Feb 26 \\
    3 & Mar 11 \\
    4 & Apr 1\\ \hline
  \end{tabular}
  \end{center}
  \caption{Demo dates.}
  \label{tab:demo-dates}
\end{table}

\begin{table*}[]
  \begin{center}
  \begin{small}
  \begin{tabular}{|c|l|l|l|c|}
    \hline
    {\bf Milestone} & {\bf Tasks} & {\bf Equipment} & {\bf Skills} & {\bf Est. Hours} \\ \hline
    1               & Develop Lego prototype & Lego & Lego building & 25\\ \cline{2-5}
                    & Develop Turtlebot prototype & Turtlebot & Turtlebot mechanics & 25\\ \cline{2-5}
                    & Test and evaluate the two & - & Analysis & 50\\ \hline
    2               & Finalise chassis design & Chassis chosen from above & - & 10\\ \cline{2-5}
                    & Integrate arm & Turtlebot arm/Lego arm & Component integration & 70\\ \cline{2-5}
                    & Integrate buttons and sensors & Buttons, sensors & Component integration & 20\\ \cline{2-5}
                    & Design and integrate scissor lift & Lego & Lego building/design & 100\\ \hline
    3               & Design and integrate robot hardware controller & Hardware components, RasPi & Electronics & 70\\ \cline{2-5}
                    & Finalise all other component choices & Whatever components necessary & Electronics, mechanics & 50\\ \hline
    4               & Maintainence & - & Debugging, testing, analysis & 80\\ \hline
                    &                           &  & {\bf Total} & 500 \\ \hline
  \end{tabular}
  \end{small}
  \caption{{\bf Robot building team} resource distribution.}
  \label{tab:rb-rd}
  \end{center}
\end{table*}

\begin{table*}[]
  \begin{center}
  \begin{small}
  \begin{tabular}{|c|l|l|l|c|}
    \hline
    {\bf Milestone} & {\bf Tasks} & {\bf Equipment} & {\bf Skills} & {\bf Est. Hours} \\ \hline
    1               & Robot moves set distances in straight line & Turtlebot/Lego-bot & ROS, Python for motors & 30\\ \cline{2-5}
                    & Robot can turn in place &  & ROS, basic physics/movement & 30\\ \cline{2-5}
                    & Begin working with sensors and readings & Sensors (input) & Analysis of sensor readings & 50 \\ \hline
    2               & Bot can locate itself & Sensors, cameras & Object recognition & 70\\ \cline{2-5}
                    & Begin using sensors for bot to stop itself & Sensors, robot & Reading analysis & 20\\ \cline{2-5}
                    & Begin software-app functionality & - & Networking & 10\\ \hline
    3               & Movement of bot complete & - & More ROS & 20\\ \cline{2-5}
                    & Obstacle avoidance & - & Machine learning & 70\\ \cline{2-5}
                    & Control of scissor lift & Access to scissor lift & - & 20\\ \cline{2-5}
                    & Map environment & - & More ML & 80\\ \hline
    4               & Complete software-app functionality & Working build of app & - & 70\\ \cline{2-5}
                    & Wiggle room for extensions & - & - & 30\\ \hline
                    &  &  & {\bf Total} & 500 \\ \hline
  \end{tabular}
  \end{small}
  \caption{{\bf Robot programming team} resource distribution.}
  \label{tab:rp-rd}
  \end{center}
\end{table*}

\begin{table*}[]
  \begin{center}
  \begin{small}
  \begin{tabular}{|c|l|l|l|c|}
    \hline
    {\bf Milestone} & {\bf Tasks} & {\bf Equipment} & {\bf Skills} & {\bf Est. Hours} \\ \hline
    1               & Create Android skeleton app & IDE & Kotlin, Android & 30\\ \cline{2-5}
                    & Set up a database management system & Firebase & XML & 30\\ \cline{2-5}
                    & Ability to send/receive data to database from app & - & Firebase with networking & 30\\ \hline
    2               & Build bot communication API & - & Work alongside robot programming team & 70\\ \cline{2-5}
                    & Integrate robot alert feature & Access to bot & Networking features & 40\\ \hline
    3               & Timetable of deliveries & - & Work alongside robot programming team & 40\\ \cline{2-5}
                    & Timetable wakeup alarm & Access to bot & Networking features & 30\\ \cline{2-5}
                    & On demand check-in request feature & - & Kotlin, Android & 40\\ \cline{2-5}
                    & Dashboard for timetable history & Memory & Memory management & 50\\ \cline{2-5}
                    & Encrypt the data transmissions & - & HTTPS & 60\\ \hline
    4               & Finalise connectivity details & - & - & 30\\ \cline{2-5}
                    & Test UI accessibility & User tests, ethical clearance & User testing & 50\\ \hline
                    &                           &  & {\bf Total} & 500 \\ \hline
  \end{tabular}
  \end{small}
  \caption{{\bf App team} resource distribution.}
  \label{tab:app-rd}
  \end{center}
\end{table*}


\subsection{Risk assessment} 

\subsubsection{Robot building}

In terms of milestone 1, the main risk would be that the robot design is not able to fulfill expected tasks (particularly carrying loads). We have decided to develop 2 prototypes as a contingency plan for this exact issue. This allows us to compare two designs and ultimately pick the one which we find to be a better choice. 25 hours for the design and building for each robot may seem like a large amount of time when one of them will be discarded. However, ultimately this counters the many more hours that we would have to spend later down the line should we find that our bot was infeasible. Having to redevelop the bot as well as the software and some features in our app would require far more than 25 hours.

For milestone 2, our team are planning to design and build a scissor lift from Lego. This should be fairly risk free. We also plan to build a hardware controller for the bot. At least one of these should be achievable and whichever is not completed for this milestone can be worked on during the period of maintenance running up to the fourth demo.

This 30 hours of maintenance between milestone 3 and 4 will actually prevent mostly any risk that should arise with individual pieces of hardware, as it can be eaten into by the hardware team if need be.

\subsubsection{Robot programming}

Having the robot move without the aid of a line on the ground is definitely a risk the robot programming team will have to keep in mind. It could prove to be more difficult than anticipated. The fact that sensors are not always 100\% accurate means that special care will have to be taken in the implementation of this. We will reduce this risk by re-planning regularly in software to make sure tolerances do not grow too much over time. 

Failure of a sensor could be catastrophic in a project where they are the centerpiece of the functionality. It is therefore of vital importance that we constantly check that they are all functioning as they should be.

The algorithms used by the robot will also be a challenge facing our robot programming team. If implementing these turns out to be more difficult than we anticipate we will refer to advice from the expert consultant in electronics and software.

\subsubsection{App}

The use of Firebase should alleviate most of the risk in terms of the building of the app as it allows an easy way to get things off the ground. There are extensive resources to aid our developers in their progress should they hit a brick wall.

The UI will have to be usable for novice users and one of the risks is that this is not the case. The app team will use user testing in order to make sure they are putting out a usable product.


\section{Group organisation}
\begin{table}[]
  \resizebox{\columnwidth}{!}{
  \begin{tabular}{lll}
    \hline
    Robot building & Robot programming & App development   \\
    \hline
    {\bf Jakub}          & {\bf Wojtek}       & {\bf Theo}              \\
    Luukas         & Ben          & Yuchen            \\
    Rebecca        & Errikos      & Ching Ling \\
    \hline
    Project management: & {\bf Michael} & \\
  \end{tabular}}
  \caption{Team splits across the group. Names in bold are key points of contact.}
  \label{tab:group-split}
\end{table}

We will split the group into {\bf three core teams}: robot building, robot software, and app development (table~\ref{tab:group-split}). Each team has a team lead (in bold), responsible for coordinating with the other groups as necessary during the development process and holding responsibility for ensuring the group is tracking to its milestones.

The project manager (PM) holds overall responsibility for keeping the project on track: primarily through helping each team plan, execute, and validate its progress against milestones. The PM is also responsible for time management; writing up reports; and helping teams with any blockers they encounter in their work.

Structuring the teams in this manner allows us to use the {\it functional} organizational structure. This has particular benefits in allowing everyone within each team to focus on developing expertise in their area, as well as improving efficiency in communication within the team.

The disadvantage of using a functional structure is that it can make communication between teams difficult. To minimize this, we have decided a key point of contact (POC), in bold, for each team: POCs will communicate with each other and the PM on key issues and when cross-team collaboration is needed --- for example, in interfacing the robot hardware with the robot software. This eases communication between teams because it means there will only need to be four people meeting at once to represent all teams, rather than all ten group members needing to be present.

{\bf Each team will meet at a minimum once a week} to discuss their progress against their milestones. Work will be done in collaboration with the PM following an Agile approach: at each weekly meeting we will follow {\bf Plan, Develop, Review} methodology to iteratively update our approach and track our progress against our milestones.

In keeping with an Agile approach, {\bf team members will give daily updates on their work} (how they are progressing, if they have any blockers, and whether they need any help from other team members) in online `standups' on Slack. Additionally the team POCs will meet once a week to make sure that any cross-team integration issues are handled, as well as to support each other if needed. 

{\bf Code-sharing} will be done exclusively through GitHub. Using version control more generally allows us to track our work over time and easily deal with any merge conflicts or other issues that may come up in doing distributed development. GitHub was chosen primarily because all members of the group are somewhat familiar with it, meaning there will be a shallower learning curve to complete our project using it. When working to demos we will put in place a {\bf code freeze} on midnight on the Sunday before the demo date to reduce risks to the system at the demo itself. 

Using GitHub also allows us to do {\bf task allocation and progress sharing} using GitHub projects. We will follow a Kanban approach, separating tasks into ``to do'', ``in progress'', and ``done'', with one board per team. Using Kanban allows for team members to clearly see what tasks need to be done before the next demo; choose to begin working on tasks they feel they are suitable for; and to identify where there may be blockers within their work (ie. cards that are spending extended time in ``to do''). Additionally, making these boards public allows for other teams to see how work is progressing in the rest of the group. Progress updates will also be discussed in weekly team meetings on a granular level and more broadly in POC meetings.

Additionally, in {\bf task allocation}, teams will decide the complexity of tasks by evaluating them using {\bf T-Shirt sizing}, where the size or difficulty of each task is graded on a scale: XS, S, M, L, XL. This will allow team members to clearly see the estimated difficulty of each task and to assess where they need to dedicate their effort. 

{\bf Communication} will be done primarily through a dedicated Slack workspace, with separate channels for each team ({\tt robot-app}, {\tt robot-building}, {\tt robot-coding}, {\tt report-writing}, and {\tt general} for cross-team discussion). Using Slack allows us to integrate other apps, for example Doodle polls to decide group meeting times; and GitHub to track push or other notifications from our project repository. 


%% Include any references in a bibliography

\bibliography{plan-refs}

\end{document} 

