%% Template for SDP report, adapted from mlp_cw2_template, 2018. 

%% Based on  LaTeX template for ICML 2017 - example_paper.tex at 
%%  https://2017.icml.cc/Conferences/2017/StyleAuthorInstructions

\documentclass{article}
\usepackage[T1]{fontenc}
\usepackage{amssymb,amsmath}
\usepackage{txfonts}
\usepackage{microtype}
\usepackage{xspace}
\xspaceaddexceptions{\%}

% Lists with less spacing between items
\usepackage{paralist}

% For figures
\usepackage{graphicx}
\usepackage{subfig} 

% For citations
\usepackage{natbib}

% For algorithms
\usepackage{algorithm}
\usepackage{algorithmic}

% the hyperref package is used to produce hyperlinks in the
% resulting PDF.  If this breaks your system, please commend out the
% following usepackage line and replace \usepackage{mlp2017} with
% \usepackage[nohyperref]{mlp2017} below.
\usepackage{hyperref}
\usepackage{url}
\urlstyle{same}

% Packages hyperref and algorithmic misbehave sometimes.  We can fix
% this with the following command.
\newcommand{\theHalgorithm}{\arabic{algorithm}}


% Set up MLP coursework style (based on ICML style)
\usepackage{mlp2018}
\mlptitlerunning{SDP Demo \demoNumber  Group (\groupNumber)}
\bibliographystyle{icml2017}


\DeclareMathOperator{\softmax}{softmax}
\DeclareMathOperator{\sigmoid}{sigmoid}
\DeclareMathOperator{\sgn}{sgn}
\DeclareMathOperator{\relu}{relu}
\DeclareMathOperator{\lrelu}{lrelu}
\DeclareMathOperator{\elu}{elu}
\DeclareMathOperator{\selu}{selu}
\DeclareMathOperator{\maxout}{maxout}







%% You probably do not need to change anything above this comment

%% REPLACE the details in the following commands with your details
\setGroupNumber{15}
\setGroupName{Detroit}
\setProductName{Tadashi}
\setDemoNumber{2}
\setLogoFileName{figs/logo-small.png}

\begin{document} 

\makeSDPTitle{Demo}

\begin{abstract}

\end{abstract} 

\section{Project plan update}
List each goal set for this demo, appended with achieved, partly achieved, not achieved.
\begin{itemize}
\item {\bf TODO write up}
\end{itemize}

Concisely summarize reasons for deviations from achievement of intended goals. 
\begin{itemize}
\item {\bf TODO write up}
\end{itemize}

Provide one-paragraph description of how group organised work towards the goals: which group members worked on which aspect? How did we ensure effective group work?
\begin{itemize}
\item {\bf TODO who did what}
\item Moving Rebecca off hardware to longer-term planning + user surveys; Yuchen taking over app development; Theo moving to UI/UX work; Ben taking lead on robot coding; 
\end{itemize}

Provide summary of budget spent so far.
\begin{itemize}
\item 1h45m of technician time
\item To build the lift: 8 + 2 + 2 + 5 = \pounds 17 (+ \pounds 37.5 of Lego)
\end{itemize}

Provide clear statement for any modification you wish to make to your goals for the next demonstration.
\begin{itemize}
\item {\bf TODO decide next goals}
\item Big change: moving user testing forward as much as possible so we can get as much done as we can
\end{itemize}

\section{Technical details}
Describe technical details of current implementation. Provide clear justification for design decisions, with brief reference to alternatives considered or explored.

\subsection{Hardware}
\begin{itemize}
\item Scissor lift designed + built 
\item Include pictures: at rest, fully extended, detail of mounts, detail of metal supports
\item Maximum weight identified + stress points identified
\item Stress points reinforced (by what?)
\item Mounting system designed + manufactured for mounting lift to Turtlebot 
\item Struts added to Turtlebot to mount lift (but mounting not finalized)
\end{itemize}

\subsection{Software}
\begin{itemize}
\item Autonomous mapping implemented
\item Autonomous movement implemented 
\item Server set up to allow communication with app
\end{itemize}

\subsection{App}
\begin{itemize}
\item Implementation of key screens: calendar, map, dashboard
\item Include pictures of each screen
\item Patient prioritization
\item Possible dashboards prepared against expected use-cases: details?
\item User tests designed to evaluate usability
\end{itemize}

\subsection{User testing}
\begin{itemize}
\item Design of survey for speaking to carehome workers
\item Analysis of how survey will affect work going forward
\item Ethical approval requested (? - check with Rebecca - will be delayed with strikes)
\end{itemize}

\section{Evaluation}
Testing:
\begin{itemize}
\item Lift team quantitative analysis:
  \begin{itemize}
  \item Can the lift get to the same height consistently? What's the margin of error to get to the max height?
  \item Is the lift stable with increasing amounts of weight?
  \item Can the lift handle off-centre loads?
  \end{itemize}
\item Coding team quantitative analysis:
  \begin{itemize}
  \item Can the robot reach its intended destination (from different starting spots?)
  \item How long does the robot take to reach its intended destination (how far does it travel vs ideal distance?)
  \item How long does the robot's battery last? Does performance degrade as battery degrades?
  \item Do the struts added to the base impede the lidar? 
  \end{itemize}
\end{itemize}
  
\section{Budget}
We need more details for this section:
\begin{itemize}
  \item Calculate how much total budget we have left. We started with 10 hours and one hour has been deducted every week since week 2. Budget started at \pounds 200. Nominal amounts (\pounds 5-10) should be included to cover misc. small items. 
  \item Luukas needs to provide detailed breakdown of costs for lift: in terms of technician time and material cost. My best guess right now is:
    \begin{itemize}
    \item 1 hour 45 minutes of technician time. 90 minutes for setting up the lift (need more details as to what this involved) + 15 minutes for Garry to drill holes in base connecting to robot.
    \item To build the lift: some Lego (likely <1kg); misc wires; WiFi USB adaptor (\pounds 5); \pounds 10 to build supports (what does this mean?) and \pounds 2 for the MDF base (12mm x 270mm x 400mm)
    \end{itemize}
  \item We need to add the following items to the budget breakdown (deducted from our \pounds 200 budget):
    \begin{itemize}
    \item Cost of misc parts (nominally \pounds 5 - 10) - to include angle brackets and screws for pegboard, misc wires etc. 
    \item Cost of pegboard to make arena. Pegboard cost is \pounds 12 per 1220 x 2440mm sheet. We need to measure how much we have used. I think we have used 6026cm squared so far but I need to double-chceck this. 
    \item Lego at \pounds 15 per kg. 
    \end{itemize}
  \item We need to add the following items to the budget breakdown, not deducted from budget:
    \begin{itemize}
    \item EV3 kit at \pounds 371.99.
    \item Turtlebot Waffle at \pounds 1275.80. 
    \item Costs are from: \url{http://www.inf.ed.ac.uk/teaching/courses/sdp/SDP2020/sdppricelist.pdf}
    \end{itemize}
\end{itemize}

We have used \pounds 12 on materials to support the lift, split as below. Additional costs for building the lift are included below but not taken out of total budget. 

\begin{tabular}{l|c}
  \hline
  {\bf Use} & {\bf Cost (\pounds)} \\
  \hline 
  Turtlebot supports & 10 \\
  MDF base for lift & 2 \\
  \hline
  Total (charged) & 12 \\
  \hline
  
\end{tabular}

We have used one hour 45 minutes of technician time, split as below:

\begin{tabular}{l|c}
  \hline
  {\bf Use} & {\bf Time cost} \\
  \hline 
  ? & 90 minutes \\ 
  Drilling holes in base for lift & 15 minutes \\
\end{tabular}

\end{document} 

