%% Template for SDP report, adapted from mlp_cw2_template, 2018. 

%% Based on  LaTeX template for ICML 2017 - example_paper.tex at 
%%  https://2017.icml.cc/Conferences/2017/StyleAuthorInstructions

\documentclass{article}
\usepackage[T1]{fontenc}
\usepackage{amssymb,amsmath}
\usepackage{txfonts}
\usepackage{microtype}
\usepackage{xspace}
\xspaceaddexceptions{\%}

% Lists with less spacing between items
\usepackage{paralist}

% For figures
\usepackage{graphicx}
\usepackage{subfig} 

% For citations
\usepackage{natbib}

% For algorithms
\usepackage{algorithm}
\usepackage{algorithmic}

% the hyperref package is used to produce hyperlinks in the
% resulting PDF.  If this breaks your system, please commend out the
% following usepackage line and replace \usepackage{mlp2017} with
% \usepackage[nohyperref]{mlp2017} below.
\usepackage{hyperref}
\usepackage{url}
\urlstyle{same}

% Packages hyperref and algorithmic misbehave sometimes.  We can fix
% this with the following command.
\newcommand{\theHalgorithm}{\arabic{algorithm}}


% Set up MLP coursework style (based on ICML style)
\usepackage{mlp2018}
\mlptitlerunning{SDP Demo \demoNumber  Group (\groupNumber)}
\bibliographystyle{icml2017}


\DeclareMathOperator{\softmax}{softmax}
\DeclareMathOperator{\sigmoid}{sigmoid}
\DeclareMathOperator{\sgn}{sgn}
\DeclareMathOperator{\relu}{relu}
\DeclareMathOperator{\lrelu}{lrelu}
\DeclareMathOperator{\elu}{elu}
\DeclareMathOperator{\selu}{selu}
\DeclareMathOperator{\maxout}{maxout}







%% You probably do not need to change anything above this comment

%% REPLACE the details in the following commands with your details
\setGroupNumber{15}
\setGroupName{Detroit}
\setProductName{Tadashi}
\setDemoNumber{1}
\setLogoFileName{figs/logo-small.png}

\begin{document} 

\makeSDPTitle{Demo}

% Previous MLP Style Title Layout working. 
% \twocolumn[
    % \mlptitle{\productName: SDP Demo \demoNumber}
    % \centerline{Group \groupNumber: \groupName}
% ]

\begin{abstract}
  {\it Tadashi} is an assistive robot to automate simple tasks within a healthcare environment: waking patients up, bringing food and water to patients at specified times, and checking on the welfare of patients when needed. 

  In this demo we will demonstrate remote control of movement and turning for the TurtleBot; as well as the ability to recognize objects in its path using the built-in Lidar sensor. We will demonstrate the Lego robot we built, although after testing we have decided to proceed with the TurtleBot. We will also demonstrate a skeleton Android app allowing for basic interaction with a database.
\end{abstract} 

\section{Project plan update} 
Our hardware team had the following goals for this demo:
\begin{itemize}
\item Build Lego and TurtleBot robots: {\bf achieved}
\item Test each robot against chosen metrics: {\bf achieved}
\item Decide which robot to use: {\bf achieved}
\end{itemize}

Our app team had the following goals for this demo:
\begin{itemize}
\item Start an Android app project in Kotlin: {\bf achieved}
\item Create five activity screens for the app: {\bf achieved}
\item Plan out and set up database schema: {\bf achieved}
\item Implement database using Firebase: {\bf achieved}
\item Show ability to connect to database (send / receive information to / from database) through app: {\bf achieved}
\end{itemize}

Our software team had the following goals for this demo:
\begin{itemize}
\item Implement moving set distances and turning: {\bf achieved}
\item Test to see if we can get inputs from overhead and internal sensors: {\bf partially achieved}
\end{itemize}

We were only partially able to achieve getting inputs from overhead and internal sensors. This was for two reasons: firstly, we chose to implement arrow key control for moving set distances and turning in the TurtleBot and this took longer than expected. Secondly, understanding how to process and use Lidar input on the Turtlebot took longer than expected because of a lack of clear documentation on its usage. This meant that we did not have time to get to using the overhead sensors before the demo deadline.

\subsection{Organisation}

In robot building,

The robot software team has been a collaborative effort using the popular agile development technique 'pair programming' - or in this case 'trio programming' - whereby at any give time one member was programming with the other two debugging, testing, or just generally brainstorming. Ben coded the original skeleton of the remote control scripts, but all 3 members contributed equally towards the final product.

In app development, Yuchen built the skeleton of the UI and implemented accessing the database; Theo improved the UI to make it more user friendly; Ching Ling has been studying Android and preparing her work on the dashboard and user fragment.

On the project management side, Michael has been off due to illness so primarily contributing to report writing / proofreading and helping with robot coding debugging. 

Robot coding and app development work has been done using GitHub to ensure effective integration of code between team members. We have continued to communicate with Slack to manage inter- and intra-group work.

\section{Technical details}


\subsection{Hardware}

Explain any construction on the hardware parts of your system, including choice and placement of sensors and actuators. Pictures should be used if appropriate (for instance, figure~\ref{fig:sample-fig}), using the \verb+\includegraphics+ environment to include an image (pdf, png, or jpg formats), ideally with informative labels added. 

\subsection{User interface}
We have set up two core activities within the app:

The {\bf login activity} authenticates the email and password provided by the user using Firebase Authentication. New users can create an account using the signup activity. After the user registers a new account, the app creates an entry for the user with Firebase Authentication and the user will be signed in automatically the next time the app starts.

The {\bf main activity} contains five fragments. For now these fragments are empty skeletons. The `patients' fragment can show a list of patient names, retrieved from Firebase Cloud Firestore, and add new patients for demonstration purposes.

In order to demonstrate user familiarity, we designed the skeleton to look like other apps on the market. We chose to use Firebase for our prototype. Firebase is much easier to use than a traditional SQL database and allows us to sync application data in real time if the database changes. For the patient list view, we chose to use `RecyclerView' rather than the traditional `ListView' for design flexibility. We chose to use Cloud Firestore to store patient details as we do not expect existing patient data to change often, only for new patients to be added. For future development we plan to use the Firebase realtime database for storing real-time patient information (if they're ok, as well as existing food / water requests).

\subsection{Software}

Explain the key details of the control and interface software developed for the project. Be clear about any packages used and the reason for choosing them. 

If you present algorithms, you can use the \verb+algorithm+ and \verb+algorithmic+ environments to format pseudocode (for instance, Algorithm~\ref{alg:example}). These require the corresponding style files, \verb+algorithm.sty+ and \verb+algorithmic.sty+ which are supplied with this package. 

\begin{algorithm}[ht]
\begin{algorithmic}
   \STATE {\bfseries Input:} data $x_i$, size $m$
   \REPEAT
   \STATE Initialize $noChange = true$.
   \FOR{$i=1$ {\bfseries to} $m-1$}
   \IF{$x_i > x_{i+1}$} 
   \STATE Swap $x_i$ and $x_{i+1}$
   \STATE $noChange = false$
   \ENDIF
   \ENDFOR
   \UNTIL{$noChange$ is $true$}
\end{algorithmic}
  \caption{Bubble Sort}
  \label{alg:example}
\end{algorithm}

\section{Evaluation}
Our main quantitative testing in this demo was in comparing the TurtleBot and Lego robots. We considered four aspects of the robot important in choosing which to use for our final product:
\begin{enumerate}
\item Rotational velocity
\item Payload
\item Battery life
\item Weight
\end{enumerate}

Information for the TurtleBot was read from the specification sheet provided by the manufacturers {\bf BEN PLEASE REFERENCE THIS DOCUMENT - I DON'T KNOW WHERE IT IS}.

Having built the Lego robot ourselves, we got information about the metrics by performing testing ourselves. Payload was measured by placing objects on the robot until the frame began to buckle, at 5kg of weight: we estimate failure of the frame at 7kg. Battery life was estimated by running the robot's motors until they slowed down to 50\% of speed. 

Having performed this testing, it is clear that the TurtleBot meets our needs much better than the Lego robot. In particular, the significant difference in maximum carrying capacity (30kg vs 7kg) alone means that we have chosen to go for the TurtleBot. As the robot will need to be capable of carrying additional loads --- food and water --- it is essential that it is able to carry these loads without being able to buckle.


\begin{table}[h]
\vskip 3mm
\begin{center}
\begin{small}
\begin{sc}
\begin{tabular}{lcc}
\hline
\abovespace\belowspace
Metric & TurtleBot & Lego \\
\hline
  Rotational velocity (rad / s) & 1.8 & 0.5 \\
  Payload (kg) & 30 & 5 in testing (est. 7) \\
  Battery life (hours) & 2 & 1.5 \\
  Weight (kg) & 1.8 & 2.5 
\end{tabular}
\end{sc}
\end{small}
\caption{Comparison of key aspects of TurtleBot and Lego robots.}
\label{tab:sample-table}
\end{center}
\vskip -3mm
\end{table}


\section{Budget}
We have not yet spent any of our budget, so there is nothing to note. 

%% Include any references in a bibliography

\bibliography{example-refs}

\end{document} 

