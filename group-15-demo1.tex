%% Template for SDP report, adapted from mlp_cw2_template, 2018. 

%% Based on  LaTeX template for ICML 2017 - example_paper.tex at 
%%  https://2017.icml.cc/Conferences/2017/StyleAuthorInstructions

\documentclass{article}
\usepackage[T1]{fontenc}
\usepackage{amssymb,amsmath}
\usepackage{txfonts}
\usepackage{microtype}
\usepackage{xspace}
\xspaceaddexceptions{\%}

% Lists with less spacing between items
\usepackage{paralist}

% For figures
\usepackage{graphicx}
\usepackage{subfig} 

% For citations
\usepackage{natbib}

% For algorithms
\usepackage{algorithm}
\usepackage{algorithmic}

% the hyperref package is used to produce hyperlinks in the
% resulting PDF.  If this breaks your system, please commend out the
% following usepackage line and replace \usepackage{mlp2017} with
% \usepackage[nohyperref]{mlp2017} below.
\usepackage{hyperref}
\usepackage{url}
\urlstyle{same}

% Packages hyperref and algorithmic misbehave sometimes.  We can fix
% this with the following command.
\newcommand{\theHalgorithm}{\arabic{algorithm}}


% Set up MLP coursework style (based on ICML style)
\usepackage{mlp2018}
\mlptitlerunning{SDP Demo \demoNumber  Group (\groupNumber)}
\bibliographystyle{icml2017}


\DeclareMathOperator{\softmax}{softmax}
\DeclareMathOperator{\sigmoid}{sigmoid}
\DeclareMathOperator{\sgn}{sgn}
\DeclareMathOperator{\relu}{relu}
\DeclareMathOperator{\lrelu}{lrelu}
\DeclareMathOperator{\elu}{elu}
\DeclareMathOperator{\selu}{selu}
\DeclareMathOperator{\maxout}{maxout}







%% You probably do not need to change anything above this comment

%% REPLACE the details in the following commands with your details
\setGroupNumber{15}
\setGroupName{Detroit}
\setProductName{Tadashi}
\setDemoNumber{1}
\setLogoFileName{figs/logo-small.png}

\begin{document} 

\makeSDPTitle{Demo}

% Previous MLP Style Title Layout working. 
% \twocolumn[
    % \mlptitle{\productName: SDP Demo \demoNumber}
    % \centerline{Group \groupNumber: \groupName}
% ]

\begin{abstract} 
The abstract should consist of one sentence describing the intended functionality of your system, followed by a few sentences (100--200 words) summarising the key advances made for this demo. This should give the reader a clear expectation of what will be demonstrated.
\end{abstract} 

\section{Project plan update} 

This section should start with your goals:
\begin{itemize}
\item List each of the goals you had set for this demo, appended with "achieved", "partly achieved" "not achieved"
\item Start an Android app project in Kotlin: {\bf achieved}
\item Create five activity screens for the app: {\bf achieved}
\item Plan out and set up database schema: {\bf achieved}
\item Implement database using Firebase: {\bf achieved}
\item Show ability to connect to database (second / receive information to / from database) through app: {\bf achieved}
\item Implement moving set distances and turning: {\bf achieved}
\item Test to see if we can get inputs from overhead and internal sensors: {\bf partially achieved}
\end{itemize}

We were only partially able to achieve getting inputs from overhead and internal sensors. This was for two reasons: firstly, we chose to implement arrow key control for moving set distances and tunring in the TurtleBot and this took longer than expected. Secondly, understanding how to process and use Lidar input on the Turtlebot took longer than expected because of a lack of clear documentation on its usage. This meant that we did not have time to get to using the overhead sensors before the demo deadline. 


In app development, Yuchen built the skeleton of the UI and implemented accessing the database; Theo improved the UI to make it more user friendly; Ching Ling has been studying Android and preparing her work on the dashboard and user fragment. 
Robot coding and app development work has been done using GitHub to ensure effective integration of code between team members. We have continued to communicate with Slack to manage inter- and intra-group work.

We have not yet spent any of our budget. 

Provide a clear statement of any modification (relative to your original plan) that you wish to make to your goals for the next demonstration.

\section{Technical details}


\subsection{Hardware}

Explain any construction on the hardware parts of your system, including choice and placement of sensors and actuators. Pictures should be used if appropriate (for instance, figure~\ref{fig:sample-fig}), using the \verb+\includegraphics+ environment to include an image (pdf, png, or jpg formats), ideally with informative labels added. 

\subsection{User interface}
We have set up two core activities within the app:

The {\bf login activity} authenticates the email and password provided by the user using Firebase Authentication. New users can create an account using the signup activity. After the user registers a new account, the app creates an entry for the user with Firebase Authentication and the user will be signed in automatically the next time the app starts.

The {\bf main activity} contains five fragments. For now these fragments are empty skeletons. The `patients' fragment can show a list of patient names, retreieved from Firebase Cloud Firestore, and add new patients for demonstration purposes.

In order to demonstrate user familiarity, we designed the skeleton to look like other apps on the market. We chose to use Firebase for our prototype. Firebase is much easier to use than a traditional SQL database and allows us to sync application data in real time if the database changes. For the patient list view, we chose to use `recyclerview' rather than a traditional `listview' to support flexibility on the design of each item. 

\subsection{Software}

Explain the key details of the control and interface software developed for the project. Be clear about any packages used and the reason for choosing them. 

If you present algorithms, you can use the \verb+algorithm+ and \verb+algorithmic+ environments to format pseudocode (for instance, Algorithm~\ref{alg:example}). These require the corresponding style files, \verb+algorithm.sty+ and \verb+algorithmic.sty+ which are supplied with this package. 

\begin{algorithm}[ht]
\begin{algorithmic}
   \STATE {\bfseries Input:} data $x_i$, size $m$
   \REPEAT
   \STATE Initialize $noChange = true$.
   \FOR{$i=1$ {\bfseries to} $m-1$}
   \IF{$x_i > x_{i+1}$} 
   \STATE Swap $x_i$ and $x_{i+1}$
   \STATE $noChange = false$
   \ENDIF
   \ENDFOR
   \UNTIL{$noChange$ is $true$}
\end{algorithmic}
  \caption{Bubble Sort}
  \label{alg:example}
\end{algorithm}

\section{Evaluation}

This section should first outline any testing methods you used (e.g. repeated runs of subsystems, data-logging, naive user testing). 

It should then present relevant quantitative results. If you are using graphs, please make sure they are properly labelled and logically illustrate the point you want to make (e.g. to compare two algorithms).


At an absolute minimum, this section should provide a table (for instance, table~\ref{tab:sample-table}, using the \verb+table+ environment) of success rates for repeated runs of the whole system (as you will only be able to show one run in the demo).

\begin{table}[h]
\vskip 3mm
\begin{center}
\begin{small}
\begin{sc}
\begin{tabular}{lcccr}
\hline
\abovespace\belowspace
Test  & Time(mins) & Errors & Success \\
\hline
\abovespace
1    & 1:30 & 0 & $\surd$ \\
2    & 3:00 & 2 & $\times$\\
3    & 2:20 & 1 & $\surd$ \\
4    & 1:50 & 1 & $\times$\\
\belowspace
5    & 2:10 & 0 & $\surd$ \\
\hline
\end{tabular}
\end{sc}
\end{small}
\caption{Results for 5 tests of the system.}
\label{tab:sample-table}
\end{center}
\vskip -3mm
\end{table}

If you need a figure or table to stretch across two columns use the \verb+figure*+ or \verb+table*+ environment instead of the \verb+figure+ or \verb+table+ environment.  Use the \verb+subfigure+ environment if you want to include multiple graphics in a single figure.

\section{Budget}
Each report should contain an actualization of the estimated total budget 
of your system.

%% Include any references in a bibliography

\bibliography{example-refs}

\end{document} 

