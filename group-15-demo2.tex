%% Template for SDP report, adapted from mlp_cw2_template, 2018. 

%% Based on  LaTeX template for ICML 2017 - example_paper.tex at 
%%  https://2017.icml.cc/Conferences/2017/StyleAuthorInstructions

\documentclass{article}
\usepackage[T1]{fontenc}
\usepackage{amssymb,amsmath}
\usepackage{txfonts}
\usepackage{microtype}
\usepackage{xspace}
\xspaceaddexceptions{\%}

% Lists with less spacing between items
\usepackage{paralist}

% For figures
\usepackage{graphicx}
\usepackage{subfig} 

% For citations
\usepackage{natbib}

% For algorithms
\usepackage{algorithm}
\usepackage{algorithmic}

% the hyperref package is used to produce hyperlinks in the
% resulting PDF.  If this breaks your system, please commend out the
% following usepackage line and replace \usepackage{mlp2017} with
% \usepackage[nohyperref]{mlp2017} below.
\usepackage{hyperref}
\usepackage{url}
\urlstyle{same}

% Packages hyperref and algorithmic misbehave sometimes.  We can fix
% this with the following command.
\newcommand{\theHalgorithm}{\arabic{algorithm}}


% Set up MLP coursework style (based on ICML style)
\usepackage{mlp2018}
\mlptitlerunning{SDP Demo \demoNumber  Group (\groupNumber)}
\bibliographystyle{icml2017}


\DeclareMathOperator{\softmax}{softmax}
\DeclareMathOperator{\sigmoid}{sigmoid}
\DeclareMathOperator{\sgn}{sgn}
\DeclareMathOperator{\relu}{relu}
\DeclareMathOperator{\lrelu}{lrelu}
\DeclareMathOperator{\elu}{elu}
\DeclareMathOperator{\selu}{selu}
\DeclareMathOperator{\maxout}{maxout}







%% You probably do not need to change anything above this comment

%% REPLACE the details in the following commands with your details
\setGroupNumber{15}
\setGroupName{Detroit}
\setProductName{Tadashi}
\setDemoNumber{2}
\setLogoFileName{figs/logo-small.png}

\begin{document} 

\makeSDPTitle{Demo}

\begin{abstract}
  Tadashi is an assistive robot for delivering food and water to residents in supported living and care home environments.

  In this demo, we demonstrate the robot mapping the arena using ROS; locating itself in the demo space; moving around the space while avoiding obstacles; recognizing when it is stuck; and autonomously moving from the starting location to a target location, and then returning to its original location. We will also demonstrate the lift being able to move weights of up to 2kg up and down in a stable manner. We will demonstrate the dashboard, calendar, and map pages of the app, and being able to send commands to the robot from the app. We will discuss our intended user testing and user surveys and how they will influence our future work.
\end{abstract} 

\section{Project plan update}
Our goals for this demo do not match the goals we originally set ourselves in the project plan for milestone 2. After demo 1, we decided to pivot our project work to focus exclusively on the food and water delivery system and dashboarding functionality. This meant changing our group goals to achieve this functionality, rather than tracking to the goals from our original plan. The goals we set ourselves for this sprint better match our intended `Agile' approach, as we will demonstrate a basic working product in this demo which will be iteratively improved for the next demo.

We have managed to achieve or partially achieve all the goals we set ourselves for this demo.
\begin{itemize}
\item Design and build scissor lift to lift up to 2kg: {\bf achieved}.
\item Identify and reinforce weak points in scissor lift: {\bf achieved}.
\item Mount lift to Turtlebot: {\bf partially achieved}.
\item Implement autonomous mapping and movement for the robot: {\bf achieved}.
\item Set up server to allow communication between the robot and the app: {\bf partially achieved}. 
\item Implement patient prioritization and key screens in app: calendar, map, dashboard: {\bf achieved}.
\item Plan out user testing and user surveys: {\bf achieved}.
\end{itemize}

We were not able to complete mounting the scissor lift to the robot as there has been a delay in ordering and receiving the officially-supported Turtlebot mount platforms. We expect to receive these within the next two weeks: in the meantime, we worked with the technicians to create a temporary platform using an MDF base.

We were only able to implement unidirectional (app $\rightarrow$ robot) communication, enough to allow the app to send instructions on where the robot should move to. This is because we had issues in sending packets between the app and server: we needed to wait until a network technician was available to help us resolve this issue. 


\subsection{Organization}
\begin{table}[]
  \begin{tabular}{c|c}
    Team & Members \\
    \hline
    Robot building & {\bf Jakub}, Luukas \\
    Robot programming & {\bf Ben}, Wojtek, Errikos \\
    App development & {\bf Yuchen}, Ching Ling \\
    Project management & {\bf Michael}  \\
    UI/UX and user testing & {\bf Theo}  \\
    Product development & {\bf Rebecca}  
  \end{tabular}
  \caption{New team splits across the group. Names in bold are key points of contact.}
  \label{tab:group-split}
\end{table}

At the beginning of this sprint, we rearranged group members across teams according to table \ref{tab:group-split}. Rebecca began the sprint in the robot building team, designing a CAD model of the lift, in case the Lego lift was too unstable and we needed to manufacture it instead. We found the Lego lift to be sufficiently stable, so Rebecca has moved to working on product development. Likewise Theo began the spring in the app development team, but after Yuchen and Ching Ling were able to implement key functionality, Theo moved to working on UI/UX to ensure the app would be usable by non-technical caregivers.

The entire team met once a week with our mentor to discuss group progress against goals. Michael met with each sub-team twice a week to track any possible blockers and identify any possible support they needed. Inter-team meetings were arranged ad-hoc between POCs for each team when collaboration was needed. 

Slack has been our main mode of communication, allowing for communication between team members and setting meetings. When needing to collaborate between teams, private DMs were set up with the POCs of each team to arrange meeting times. As with previous sprints, we have used GitHub to allow collaboration when writing code for the robot and the app. In writing the app, each member of the team worked on a feature branch and once work was complete, the team met to merge these branches and resolve any merge conflicts together. 

Budget spending so far consists of \pounds 1787.43 on monetary costs that come from the starting kit, so not deducted from total budget (the Turtlebot, and parts needed to make the lift); \pounds 22.59 spent on monetary costs to build the lift and prepare the arena; and 1.75 hours of technician time. This leaves us with \pounds 177.41 remaining in our budget and 5.25 hours of technician time. 

We have changed our intended goals from the original project plan, so we have decided new goals for demo 3 as below:
\begin{itemize}
\item Begin testing app with non-technical users to gain insights on accessibility and ease of use; and make changes to app justified against results of testing.
\item Complete user interviews with caregivers to assess demand for project features and implement new features based on caregiver requests. \footnote{We intend for this to be done by demo 3; however, due to UCU strike action it is possible that ethical approval will be delayed which will delay our ability to do interviews. This is outside of our control.}
\item Physically integrate lift mechanism into robot body, and integrate lift control into robot control software.
\item Add sensor to lift to determine whether items have been removed. 
\item Refine robot problem-solving when stuck by tuning mapping sensitivity and obstacle avoidance logic.
\item Implement bi-directional communication between the app and robot.
\end{itemize}

\section{Technical details}
Describe technical details of current implementation. Provide clear justification for design decisions, with brief reference to alternatives considered or explored.

\subsection{Hardware}
\begin{itemize}
\item Scissor lift designed + built 
\item Include pictures: at rest, fully extended, detail of mounts, detail of metal supports
\item Maximum weight identified + stress points identified
\item Stress points reinforced (by what?)
\item Mounting system designed + manufactured for mounting lift to Turtlebot 
\item Struts added to Turtlebot to mount lift (but mounting not finalized)
\end{itemize}

\subsection{Software}
\begin{itemize}
\item Autonomous mapping implemented
\item Autonomous movement implemented 
\item Server set up to allow communication with app
\end{itemize}

\subsection{App}
\begin{itemize}
\item Implementation of key screens: calendar, map, dashboard
\item Include pictures of each screen
\item Patient prioritization
\item Possible dashboards prepared against expected use-cases: details?
\item User tests designed to evaluate usability
\end{itemize}

\subsection{User testing}
\begin{itemize}
\item Design of survey for speaking to carehome workers
\item Analysis of how survey will affect work going forward
\item Ethical approval requested (? - check with Rebecca - will be delayed with strikes)
\end{itemize}

\section{Evaluation}
Testing:
\begin{itemize}
\item Lift team quantitative analysis:
  \begin{itemize}
  \item Can the lift get to the same height consistently? What's the margin of error to get to the max height?
  \item Is the lift stable with increasing amounts of weight?
  \item Can the lift handle off-centre loads?
  \end{itemize}
\item Coding team quantitative analysis:
  \begin{itemize}
  \item Can the robot reach its intended destination (from different starting spots?)
  \item How long does the robot take to reach its intended destination (how far does it travel vs ideal distance?)
  \item How long does the robot's battery last? Does performance degrade as battery degrades?
  \item Do the struts added to the base impede the lidar? 
  \end{itemize}
\end{itemize}
  
\section{Budget}
The bulk of our budget leading up to this demo has been spent on building, reinforcing, and mounting the lift. We have spent a total of \pounds 22.59 from our monetary budget and 1.75 hours of technician time. Detailed breakdowns of costs are provided in tables \ref{tab:non-budget-cost}, \ref{tab:budget-cost-monetary}, and \ref{tab:budget-cost-non-monetary}.

\begin{table}[h]
\begin{center}
    \resizebox{\columnwidth}{!}{
  \begin{tabular}{lllll}
    {\bf Item} & {\bf Units} & {\bf Cost (\pounds\ per unit)} & {\bf Total cost (\pounds)} & {\bf Use} \\
    \hline
    Turtlebot Waffle & 1 & 1275.80 & 1275.80 & Robot \\
    EV3 brick & 1 & 247.19 & 247.19 & Lift \\
    EV3 battery & 1 & 86.39 & 86.39 & Lift \\
    EV3 battery charger & 1 & 32.39 & 32.39 & Lift \\
    EV3 large motor & 2 & 32.39 & 64.78 & Lift \\
    EV3 touch sensor & 2 & 19.19 & 38.38 & Lift \\
    WiFi USB adaptor & 1 & 5.00 & 5.00 & Lift \\
    Lego & 2.5kg & 15.00 & 37.50 & Lift \\
    \hline \hline
         &       & Overall cost (\pounds) & 1787.43
  \end{tabular}}
\caption{Non-budgeted monetary costs at demo \demoNumber.}
\label{tab:non-budget-cost}
\end{center}
\end{table}

\begin{table}[h]
\begin{center}
  \resizebox{\columnwidth}{!}{
  \begin{tabular}{lllll}
    {\bf Item} & {\bf Total cost (\pounds)} & {\bf Use} \\
    \hline
    Turtlebot supports  & 8.00 & Lift \\
    MDF sheet (12mm x 270mm x 400mm) & 2.00 & Lift \\
    Lift support rods & 5.00 & Lift \\
    Pegboard ($0.62 \text{m}^2$ at \pounds 12 per $2.88 \text{m}^2$)& 2.59 & Arena \\
    Misc (wires, screws, angle brackets) & 5.00 & Lift, arena \\
    \hline
    \hline
     Budget spent (\pounds) & 22.59 \\
     Budget remaining (\pounds) & 177.41   
  \end{tabular}}
\caption{Budgeted monetary costs at demo \demoNumber.}
\label{tab:budget-cost-monetary}
\end{center}
\end{table}

\begin{table}[h]
\begin{center}
  \resizebox{\columnwidth}{!}{
  \begin{tabular}{lll}
    {\bf Purpose} & {\bf Hours spent} & {\bf Use} \\
    \hline
    Manufacturing lift support rods & 1 & Lift \\
    Manufacturing lift base (MDF) & 0.5 & Lift \\
    Drilling holes in lift base & 0.25 & Lift \\
    \hline
    \hline
    Overall technician hours spent & 1.75 & \\
    Total technician hours remaining & 5.25
  \end{tabular}}
\caption{Budgeted technician time at demo \demoNumber. Note that hours remaining includes subtracting any hours lost due to not being used in previous weeks.}
\label{tab:budget-cost-non-monetary}
\end{center}
\end{table}

\end{document} 

